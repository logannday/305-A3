\documentclass{article}
\title{Algorithms Assignment 3}
\author{Logan Day}
\date{\today}
\usepackage[shortlabels]{enumitem}

\begin{document}
\maketitle
\section{}

\begin{enumerate}[a.]

    \item

        The two algorithms will produce different results when the input array
        is $[1, 2, 3]$. In the case of BUILD-MAX-HEAP', first 1 is inserted
        as the root of the array, then 2 is inserted and swapped with one. at
        this point the heap looks like $[2, 1]$. Finally, Three is inserted,
        and swapped with two, resulting in the array $[3, 1, 2]$. In the case
        of BUILD-MAX-HEAP, the operations are done directly on the array, instead
        of adding values to a new array. First, Max heapify is called on the
        root, as it is the first non leaf node. Because 3 is the larger of the
        two child nodes, 1 is swapped with three, resulting in the array $[3,
        2, 1]$. This counter example confirms that the two algorithms do not
        produce the same result on all imputs. 

    \item

        BUILDMAXHEAP operates in $\Theta(nlg{n})$. Because the number of
        calls to MAX-HEAP-INSERT is bounded by n, the overall runtime of the
        algorithm is n x the cost MAX-HEAP-INSERT. The cost of each max heap insert
        call is $\log{d}$, where d is the depth of each node being inserted, as
        each call to MAX-HEAP-INSERT bubbles the inserted node up the tree until
        it is smaller than it's parent node. In the worst case, each node will be
        the new largest node in the tree, and there for will have to be bubbled
        up all the way up the tree until it replaces the root. Each node is inserted
        at a height of $\lg{n}$, where n is the index of the node, so the total
        number of swaps can be calculated by $\sum_2^n{ \lfloor \lg{n} \rfloor }$. 
        This is equivalent to  $\sum_1^n{ \lfloor \lg{n} \rfloor } -  \lfloor \lg{n} \rfloor 
        = \theta(n\lg{n}) - \lfloor \lg{n} \rfloor = \theta(n\lg{n})$

\end{enumerate}

\section{}

    General note for this one: each node has to bubble all the way down to the level
    it started at. at least half the nodes are on the bottom level, so the runtime
    of the algorithm is dominated by the bottom nodes.

    Because the number of calls to max heapify is bounded by n, the overall
    runtime of the algorithm is bounded by n x the cost of Max heapify. In
    either of the cases, the max heap is created in O(n) time, and
    sequentially, each the root of the heap and the last value of the heap are
    swapped, and the last value is sifted down into a position that maintains
    the heap property.

\section{}
    Note to self for tomorrow: the worst case for the top node is o(n), then 
    o(n-1), o(n-2) etc. Find a sum to calculate this and youre good. Also mention 
    the for loop in the beginning of the thing. 

    My algorithm: For each node in the heap, recursively check that the value
    of the node is greater than the value of each node in it's left subtree,
    and less than the value of each node in it's right subtree. This is
    implemented as return checkLeftSubtree(leftChild, rootValue) &&
    checkRightSubtree(rightChild, rootValue). The implementation of the inner
    functions follows. CheckLeftSubtree(node n, int rootValue): If the n is
    null, return true. If n's value is greater than the root node value, return
    false. If neither of these calls return true, return
    (checkLeftSubtree(n.leftChild, rootValue) && checkLeftSubtree(n.rightChild,
    rootValue). CheckRightSubtree is the same, but with the greater than replaced
    by less than. In the best case, one of the checks fails in the first iteration,
    and the runtime of the algorithm is $\theta$ of one in the result. In the worst
    case, each node in the tree

\end{document}
